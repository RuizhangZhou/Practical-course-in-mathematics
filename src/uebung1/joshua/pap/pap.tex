% PREAMBLE
\documentclass[preview, border=10pt, ngerman]{standalone}
\usepackage{babel}

% PACKAGES
% Math
\usepackage{amsmath}
\usepackage{amssymb}
\usepackage{mathtools}
\usepackage{stmaryrd}
\usepackage{dsfont}
% Typesetting
\usepackage{fontspec}
\usepackage{microtype}
% Layout
\usepackage[margin=0cm]{geometry}
% Miscellanous
\usepackage{array} % for advanced tabular etc.
\usepackage{listings} % for code inclusion
\usepackage{csquotes} % for \enquote{}
\usepackage{enumitem} % for \begin{enumerate}[label=\alph*]
\usepackage{tikz} % for \begin{tikzpicture}
\usepackage{forest} % for trees
\usepackage{hyperref} % for clickable reflinks

\DeclarePairedDelimiter{\ceil}{\lceil}{\rceil}
\DeclarePairedDelimiter{\floor}{\lfloor}{\rfloor}
\DeclarePairedDelimiter{\abs}{\lvert}{\rvert}
\DeclarePairedDelimiter{\lvalue}{\llbracket}{\rrbracket}
\DeclarePairedDelimiter{\dia}{\langle}{\rangle}
\DeclarePairedDelimiter{\paren}{\lparen}{\rparen}

\usetikzlibrary{matrix,shapes,arrows,positioning,chains,calc}
\tikzset{
    decision/.style={
        diamond,
        draw,
        text width=10em,
        inner sep=0em,
        aspect=2,
        anchor=center,
    },
    block/.style={
        rectangle,
        draw,
        text width=10em,
        text centered,
        anchor=center,
    },
    subprogram/.style={
        rectangle,
        draw,
        inner xsep=1em,
        text width=9em,
        text centered,
        anchor=center,
        path picture={\draw (path picture bounding box.north west) -- ++ (0.3,0) |- (path picture bounding box.south west); \draw (path picture bounding box.north east) -- ++ (-0.3,0) |- (path picture bounding box.south east);}
    },
    start/.style={
        rectangle,
        draw,
        text width=10em,
        text centered,
        rounded corners,
        anchor=center,
    },
    io/.style={
        trapezium,
        draw,
        trapezium left angle=60,
        trapezium right angle=120,
        text width=9em,
        text centered,
        anchor=center,
    },
    cloud/.style={
        draw,
        ellipse,
        minimum height=2em,
        anchor=center,
    },
    descr/.style={
        fill=white,
        inner sep=2.5pt
    },
    connector/.style={
        -latex,
        font=\scriptsize
    },
    rectangle connector/.style={
        connector,
        to path={(\tikztostart) -- ++(#1,0pt) \tikztonodes |- (\tikztotarget) },
        pos=0.5
    },
    rectangle connector/.default=-2cm,
    straight connector/.style={
        connector,
        to path=--(\tikztotarget) \tikztonodes
    }
}
\begin{document}
    Vorab: Das Eingabepolynom wird als \(f(x) = ax^2 + bx + c\) geschrieben. \(n_0\) ist die Anzahl der Nullstellen. \\
    \begin{tikzpicture}
        \matrix [matrix of nodes, column  sep=2cm,row  sep=8mm, align=center, nodes in empty cells] (m) {
            |[start]| {Start} & \\
            |[io]| {Eingabe: \(a, b, c\)} \\
            |[block]| {\(n_0 = 0\) \\ \(x_{1, \text{Re}} = 0\) \\ \(x_{2, \text{Re}} = 0\) \\ \(x_{1,\text{Im}} = 0\) \\ \(x_{2,\text{Im}} = 0\)} \\
            |[decision]| {Ist \(p(x)\) konstant? (Gilt \(a = 0\) und \(b = 0\)?)} & & |[decision]| {Ist \(p(x)\) linear? (Gilt \(a = 0\)?)} & |[block]| {\(p = \frac{b}{a}, q = \frac{c}{a}\)} & |[subprogram]| {\(s = \sqrt{\texttt{DBL\_MAX}}\)} & |[subprogram]| {\(p' = \abs{p}\)} & |[decision]| {Gilt \(\frac{p'}{2} > s\)?} & |[block]| {\(d = 0.25 - \frac{\frac{q}{p}}{p}\)} \\
            |[decision]| {Ist \(c = 0\)?} & & |[block]| {\(n_0 = 1, x_{1,\text{Re}} = -\frac{c}{b}\)} & |[subprogram]| {\(w = \sqrt{-d}\)} & |[decision]| {Ist \(d = 0\)?} & |[decision]| {Ist \(d \geqslant 0\)?} & |[block]| {\(d = \paren*{\frac{p}{2} \cdot \frac{p}{2}} - q\)} & |[subprogram]| {\(v = \sqrt{d}\)} \\
            |[block]| {\(n_0 = \texttt{infinity}\)} & & & |[block]| {\(n_0 = 2\) \\ \(x_{1, \text{Re}} = -\frac{p}{2}\) \\ \(x_{2, \text{Re}} = -\frac{p}{2}\) \\ \(x_{1,\text{Im}} = w\) \\ \(x_{2,\text{Im}} = -w\)} & |[block]| {\(n_0 = 1, x_{1,\text{Re}} = -\frac{p}{2} \)} & |[subprogram]| {\(w = \sqrt{d}\)} & & |[block]| {\(w = p' \cdot v\)} \\
            & & & & & |[block]| {\(n_0 = 2\)} \\
            & & & & & |[decision]| {Ist \(p > 0\)?} & |[block]| {\(x_{2, \text{Re}} = -\frac{p}{2} - w\)} \\
            & & & & & |[block]| {\(x_{1,\text{Re}} = -\frac{p}{2} + w\)} & |[block]| {\(x_{1,\text{Re}} = \frac{q}{x_{2, \text{Re}}}\)} \\
            & & & & & |[block]| {\(x_{2, \text{Re}} = \frac{q}{x_{1,\text{Re}}}\)} \\
            |[io]| {Ausgabe: \(n_0, x_{1,\text{Re}}, x_{2,\text{Re}}\), \(x_{1,\text{Im}}, x_{2,\text{Im}}\)} \\
            |[start]| {Ende} \\
        };
    
        \draw [->] (m-1-1.south) -- (m-2-1.north);
        \draw [->] (m-2-1.south) -- (m-3-1.north);
        \draw [->] (m-3-1.south) -- (m-4-1.north);
        \draw [->] (m-4-1.east) -- ++ (0.5, 0) node [above] {Nein} -- (m-4-3.west);
        \draw [->] (m-4-1.south) -- ++ (0, -0.2) node [right] {Ja} -- (m-5-1.north);
        %\draw [->] (m-5-1.east) -- ++ (0.5, 0) node [above] {Nein} -- (m-5-2.west);
        \draw [->] (m-5-1.south) -- ++ (0, -0.2) node [right] {Ja} -- (m-6-1.north);
        
        \draw [->] (m-4-3.east) -- ++ (0.5, 0) node [above] {Nein} -- (m-4-4.west);
        \draw [->] (m-4-3.south) -- ++ (0, -0.2) node [right] {Ja} -- (m-5-3.north);
        
        \draw [->] (m-4-4.east) -- (m-4-5.west);
        \draw [->] (m-4-5.east) -- (m-4-6.west);
        \draw [->] (m-4-6.east) -- (m-4-7.west);
        
        \draw [->] (m-4-7.east) -- ++ (0.5, 0) node [above] {Ja} -- (m-4-8.west);
        \draw [->] (m-4-7.south) -- ++ (0, -0.2) node [right] {Nein} -- (m-5-7.north);
        
        \draw [->] (m-5-7.west) -- (m-5-6.east);
        
        \draw [->] (m-5-6.west) -- ++ (-0.5, 0) node [above] {Nein} -- (m-5-5.east);
        \draw [->] (m-5-6.south) -- ++ (0, -0.2) node [right] {Ja} -- (m-6-6.north);
        
        \draw [->] (m-5-5.west) -- ++ (-0.5, 0) node [above] {Nein} -- (m-5-4.east);
        \draw [->] (m-5-5.south) -- ++ (0, -0.2) node [right] {Ja} -- (m-6-5.north);
        
        \draw [->] (m-5-4.south) -- (m-6-4.north);
        
        \path (m-6-6.south) -- (m-7-6.north) coordinate [midway] (m76a);
        \draw [->] (m-6-6.south) -- (m76a);
        \draw [->] (m76a) -- (m-7-6.north);
        
        \draw [->] (m-4-8.south) -- (m-5-8.north);
        \draw [->] (m-5-8.south) -- (m-6-8.north);
        \draw [->] (m-6-8.south) |- (m76a);
        
        \draw [->] (m-7-6.south) -- (m-8-6.north);
        
        \draw [->] (m-8-6.east) -- ++ (0.5, 0) node [above] {Ja} -- (m-8-7.west);
        \draw [->] (m-8-6.south) -- ++ (0, -0.2) node [right] {Nein} -- (m-9-6.north);
        
        \draw [->] (m-9-6.south) -- (m-10-6.north);
        
        \draw [->] (m-8-7.south) -- (m-9-7.north);
        
        \path let \p1 = (m-11-1.north), \p2 = (m-10-6.south) in coordinate (m101s) at (\x1,\y2);
        \path (m101s) -- (m-11-1.north) coordinate [midway] (m101m);
        \draw [->] (m101m) -- (m-11-1.north);
        
        \path let \p1 = (m-5-2.south), \p2 = (m101m) in coordinate (m102m) at (\x1,\y2);
        \path let \p1 = (m-5-3.south), \p2 = (m101m) in coordinate (m103m) at (\x1,\y2);
        \path let \p1 = (m-6-4.south), \p2 = (m101m) in coordinate (m104m) at (\x1,\y2);
        \path let \p1 = (m-6-5.south), \p2 = (m101m) in coordinate (m105m) at (\x1,\y2);
        \path let \p1 = (m-10-6.south), \p2 = (m101m) in coordinate (m106m) at (\x1,\y2);
        
        \draw [->] (m-9-7.south) |- (m106m);
        \draw [->] (m106m) -- (m105m);
        \draw [->] (m105m) -- (m104m);
        \draw [->] (m104m) -- (m103m);
        \draw [->] (m103m) -- (m102m);
        \draw [->] (m102m) -- (m101m);
        
        \draw [->] (m-6-1.south) -- (m101m);
        \draw [->] (m-5-1.east) -- ++ (0.5, 0) node [above] {Nein} -| (m102m);
        %\draw [->] (m-5-2.south) -- (m102m);
        \draw [->] (m-5-3.south) -- (m103m);
        \draw [->] (m-6-4.south) -- (m104m);
        \draw [->] (m-6-5.south) -- (m105m);
        \draw [->] (m-10-6.south) -- (m106m);
        
        \draw [->] let \p1 = (m-5-2.south), \p2 = (m101m) in (m-5-2.south) -- (\x1,\y2);
        
        \draw [->] (m-11-1.south) -- (m-12-1.north);
        
    \end{tikzpicture}
\end{document}